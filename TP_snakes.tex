\documentclass[10pt,a4paper]{article}
\usepackage[a4paper]{geometry}
\usepackage[latin1]{inputenc}
\usepackage[francais]{babel}%english
\usepackage{amsmath}
\usepackage{amsfonts}
\usepackage{amssymb}
\usepackage{mathtools}
\usepackage{makeidx}
\usepackage{graphicx}
\usepackage{caption}
\usepackage{subcaption}
\usepackage{color}
\usepackage{import}


\usepackage[T1]{fontenc}

\usepackage{float}

\usepackage{pdfpages}

\usepackage{epstopdf}

\usepackage{hyperref}

\title{TP contours d\'{e}formables : les snakes}
\author{} %, \href{mailto:matthew.ozon@cpe.fr}{matthew.ozon@cpe.fr}
\date{\today}

%begining of the document
\begin{document}
\maketitle

\section{Introduction} 

\subsection{Objectif}
Ce TP a pour objectif de vous faire appr\'{e}hender une technique de d\'{e}tection par contours actifs : les SNAKES. Les composantes de ce TP sont les suivantes :
\begin{itemize}
	\item[1] Impl\'{e}mentation C/C++ d'un algorithme de snakes
	\item[2] Caract\'{e}risation de l'algorithme en fonction des param\`{e}tres
	\item[3] Modification du mod\`{e}le
\end{itemize}

\subsection{\`{A} rendre}

\noindent\textbf{Consigne \`{a} observer} Le TP se fait en C/C++ pour la partie algorithmique et sous \textit{octave} pour la visualisation de l'\'{e}volution des snakes. Aucun IDE n'est impos\'{e}, cependant, si vous en utiliser un, nous conseillons fortement \textit{Qtcreator} (le fichier cmake est fourni).

\noindent{\bf Consignes pour le rendu \`{a} remettre avant le 15 mars 23h55}

\`{A} rendre sur le \href{https://e-campus.cpe.fr/mod/assignment/view.php?id=16984}{d\'{e}p\^{o}t} sur e-campus une archive nomm\'{e}e : TP\_SNAKE\_NOM1\_NOM2.tar. Pour g\'{e}n\'{e}rer cette archive, utilisez la commande : 
\begin{center} 
	\textit{tar -cvf TP\_SNAKE\_NOM1\_NOM2.tar fichier1 fichier 2... fichierN}
\end{center}
qui g\'{e}n\`{e}re l'archive TP\_SNAKE\_NOM1\_NOM2.tar contenant les fichiers fichier1, fichier2 jusqu'\`a fichierN. Le contenu de l'archive est le suivant : 
\begin{itemize}
	\item un compte rendu qui explique ce que fait votre code en 5 ligne, et une page pour expliquer vos experiences
	\item votre code pour calculer l'evolution du snake (il faut que le code compile) qui comporte les fichier sources, un fichier pour compiler (makefile ou cmake) et les resources (images).
	\item un script pour afficher vos resultats.
\end{itemize}

Dans le cas ou votre script affiche des figures,  n'oubliez pas de mettre des titres, des l\'{e}gendes, les labels, les conditions d'ex\'{e}cution, les param\`{e}tres de l'algorithme... tout ce qui permet de comprendre la figure car elle ne parle pas d'elle m\^{e}me (on voit clairement sur la figure... \`{a} bannir car le lecteur ne voit pas la m\^{e}me chose que vous).

Si votre script affiche des r\'{e}sultats dans la console, vous pouvez utiliser la commande :
\begin{center}
 \textit{printf('mon texte avec des variables \%i \%f', var1, var2)}
\end{center}
qui vous permet d'afficher proprement du texte dans la console avec des valeurs (comme la fonction \textit{printf} en C). Encore une fois, les valeurs ne parlent pas d'elle m\^{e}mes. Elles ont besoin d'avoir une interpr\'{e}tation. Est-ce qu'elles appartiennent \`{a} un domaine sp\'{e}cifique, est-ce des valeurs grandes/petites par rapport \`{a} quelque-chose, ont elles des unit\'{e}s, est-ce coh\'{e}rent avec ce \`{a} quoi vous vous attendiez et pourquoi, est-ce une valeur al\'{e}atoire ou d\'{e}terministe, est-ce un param\`{e}tres, est-ce une valeur moyenne, y a-t-il besoin d'autant de chiffres apr\`{e}s la virgule (significatif)... bref, expliciter les valeurs.

~


\noindent{\bf Remarque : } touts les codes et scripts doivent \^{e}tre comment\'{e}s.




\section{Rappel de cours}
\subsection{Formulation du probleme}
\paragraph{Cahier des charges} On souhaite segmenter des objets dans une image en les entourant avec une courbe param\'{e}tr\'{e}e ferm\'{e}e. On va donc utiliser les outils math\'{e}matiques li\'{e}s aux courbes param\'{e}triques. On rappelle qu'une courbe $(\Gamma)$ peut se param\'{e}trer par l'application $\gamma$ :
\begin{align*}
	\gamma :  [0,1] &\rightarrow \mathbb{R}^2\\
			s &\mapsto \begin{pmatrix}\gamma_x(s)\\ \gamma_y(s)\end{pmatrix}
\end{align*}
ou $\gamma_x(s)$ et $\gamma_y(s)$ sont les coordonn\'{e}es de la courbe. 


\paragraph{Une courbe avec de bonnes propri\'{e}t\'{e}s} Deux propri\'{e}t\'{e}s sont d\'{e}sir\'{e}es pour une courbe : la continuit\'{e} et la r\'{e}gularit\'{e} (lisse). La continuit\'{e} se traduit par une norme de la d\'{e}riv\'{e}e qui doit \^{e}tre fini et la r\'{e}gularit\'{e} impose des d\'{e}riv\'{e}es secondes de normes finies. Pour d\'{e}finir une \'{e}nergie p\'{e}nalisant les mauvaises configurations de $\Gamma$, on ajoute toutes les petites contributions de la norme de la d\'{e}riv\'{e}e et de la d\'{e}riv\'{e}e seconde sur le long de la courbe en pond\'{e}rant les deux termes : 
\begin{displaymath}
	\mathcal{E}_{\text{int}}(\gamma) = \int_{0}^{1} \frac{1}{2} \left(\lambda_1 \|\gamma'(s) \|_{2}^{2} + \lambda_2 \|\gamma''(s) \|_{2}^{2}\right) \mathrm{d}s
\end{displaymath}
avec $\lambda_1$ et $\lambda_2$ des r\'{e}els positifs modulant les effets de chaque terme.

\paragraph{Collage aux contours...} Une autre bonne propri\'{e}t\'{e} de la courbe est quelle soit coll\'{e}e aux contours des objets que l'on souhaite segmenter \textbf{dans l'image I}. Pour cela, on cherche un crit\`{e}re qui se minimise lorsque la courbe vient se positionn\'{e}e sur le contour de l'objet \`{a} segment\'{e}. Plusieurs solutions sont possibles, parmi lesquelles on trouve : 
\begin{itemize}
	\item[1] la norme du gradient de l'image (son oppos\'{e}) qui passe par un extremum sur les changements brusque d'intensit\'{e}s qui ont tendances \`{a} repr\'{e}senter des contours,
	\item[2] l'image elle m\^{e}me dans le cas de dessins : contours lin\'{e}aires fins fonc\'{e}s sur fond clair,
	\item[3] la transform\'{e}e de distance dans le cas des images binaires : le gradient \'{e}tant nul loin des contours (2 pixels) la courbe ne sera pas attir\'{e}e par les zones de transitions, cependant, la transform\'{e}e de distance est \'{e}lev\'{e}e lorsqu'on est loin des contours d'un objet et nulle dans l'objet (point \~{a} prendre en compte pour ne pas laisser le snake se contracter en une singularit\'{e}).
\end{itemize}
Pour le premier cas, on d\'{e}finit l'\'{e}nergie externe comme la contribution de touts les lieux de la courbe pond\'{e}r\'{e}s par la valeur de la norme du gradient au carr\'{e}e (circulation le long de la courbe), et on l'\'{e}crit : 
\begin{displaymath}
	\mathcal{E}_{\text{ext}}(\gamma) = \int_{0}^{1} - \lambda_3    \left( \frac{\partial \text{I}}{\partial x}^{2} + \frac{\partial \text{I}}{\partial y} ^{2} \right)(\gamma ) \mathrm{d}s
\end{displaymath}
avec $\lambda_3$ un r\'{e}el positif pond\'{e}rant la contribution de l'\'{e}nergie li\'{e}e \`{a} l'image.

Au total, nous avons la formulation suivante de l'\'{e}nergie : 
\begin{eqnarray}\label{fonctionnelle}
	\mathcal{E}_{\text{tot}}(\gamma) =& \int_{0}^{1} \frac{1}{2} \left(\lambda_1 \|\gamma'(s) \|_{2}^{2} + \lambda_2 \|\gamma''(s) \|_{2}^{2}\right)  - \lambda_3    \left( \frac{\partial \text{I}}{\partial x}^{2} + \frac{\partial \text{I}}{\partial y}^{2} \right)(\gamma) \mathrm{d}s\\
	= & \int_{0}^{1} \text{E}_{\text{int}}(\gamma',\gamma'') + \text{E}_{\text{ext}}(\gamma) \mathrm{d}s
\end{eqnarray}
Pour minimiser $\mathcal{E}_{\text{tot}}$, nous allons avoir recours aux \'{e}quations d'\href{https://e-campus.cpe.fr/file.php/2552/Cours\_Meca.pdf}{Euler-Lagrange}.



\subsection{Minimisation de fonctionnelle par equation d'Euler-Lagrange}
On entend par fonctionnelle la generalisation des fonctions aux domaine des fonction. Par exemple, si $f$ est une fonction de $\mathcal{C}^\infty(\mathbb{R}^n,\mathbb{R}^m)$ l'espace des fonction continue de $\mathbb{R}^n$ dans $\mathbb{R}^m$, alors $G(f,f',f'')=\|f\|+\|f'\|+\|f''\|$ est une fonctionnelle. Dans le cas d'une formulation int\'{e}grale~\eqref{fonctionnelle}, comme la fonctionnelle $\mathcal{E}_{\text{tot}}$, les \'{e}quations d'Euler-Lagrange donnent un crit\`{e}re pour minimiser en fonction de la courbe $\Gamma$ qui est :
\begin{displaymath}
	\gamma\in\underset{\gamma\in\mathcal{C}^{\infty}([0,1],\mathbb{R}^2)}{\arg\min} \mathcal{E}_{\text{tot}}(\gamma) \quad\text{    tel que     }\quad \frac{\partial E_{\text{tot}}}{\partial \gamma} -\frac{\mathrm{d}}{\mathrm{d}s}\left(\frac{\partial E_{\text{tot}}}{\partial \gamma'}\right) + \frac{\mathrm{d}^2}{\mathrm{d}s^2}\left(\frac{\partial E_{\text{tot}}}{\partial \gamma''}\right) = 0
\end{displaymath}
avec $E_{\text{tot}} = E_{\text{int}} + E_{\text{ext}}$. En traduisant directement l'\'{e}quation pr\'{e}c\'{e}dente, on trouve l'\'{e}quation d'\'{e}quilibre suivante :
\begin{displaymath}
	\forall s\in[0,1],\quad 0 = \lambda_1 \gamma''(s) - \lambda_2 \gamma^{(4)}(s) + \lambda_3 \nabla(\|\nabla \text{I}\|^2)
\end{displaymath}
Avec cette formulation, on a une condition sur la courbe lorsqu'elle est d\'{e}j\`{a} sur le contour de l'objet, lorsqu'elle est stationnaire. On ne peut pas faire bouger la courbe \`{a} partir de cette \'{e}quation. On introduit alors un autre param\`{e}tre, le temps $t$, pour \'{e}crire une \'{e}quation d'\'{e}volution : 
\begin{equation}\label{temporelContinu}
	\forall s\in[0,1], t\in\mathbb{R}^{+},\quad \frac{\partial \gamma}{\partial t}(s,t) = \lambda_1 \frac{\partial^2 \gamma}{\partial s^2}(s,t) - \lambda_2 \frac{\partial^4 \gamma}{\partial s^4}(s,t) + \lambda_3 \nabla(\|\nabla \text{I}\|^2)(\gamma(s,t))
\end{equation}
On dispose donc de deux \'{e}quations scalaires pour d\'{e}crire l'\'{e}volution de la courbe, une pour chacune de ses composantes.

\subsection{Discr\'{e}tisation et \'{e}volution temporelle}
L'\'{e}quation~\eqref{temporelContinu} \'{e}tant sous sa forme continue, il faut la discr\'{e}tiser pour la rendre utilisable. Il faut donc faire deux typse de discr\'{e}tisation, une temporelle et une spatiale. On notera avec un exposant $k$ les it\'{e}rations temporelles et un indice $n$ les rep\`{e}res spatiaux. On utilise des notations vectorielles pour les coordonn\'{e}es de la courbe, $\gamma_{x}^{k}$ et $\gamma_{y}^{k}$. Ces vecteurs regroupent toutes les coordonn\'{e}es de la courbe discr\`{e}te \`{a} l'it\'{e}ration $k$ : 
\begin{displaymath}
	\gamma_{x}^{k} = (\gamma_{x,0}^{k},\gamma_{x,1}^{k}\ldots,\gamma_{x,N-1}^{k})^{t}, \quad\text{    et    }\quad \gamma_{y}^{k} = (\gamma_{y,0}^{k},\gamma_{y,1}^{k}\ldots,\gamma_{y,N-1}^{k})^{t}
\end{displaymath}
avec $N$ le nombre de point de discr\'{e}tisation de la courbe. \`{A} partir de ces vecteurs, on peut \'{e}crire de fa\c{}on matricielle les op\'{e}rations de d\'{e}rivations. On note la d\'{e}riv\'{e}e seconde et quatri\`{e}me respectivement $D_2$ et $D_4$. 
\begin{displaymath}
	D_2 = \begin{pmatrix}
			 -2 & 1  & 0 & \ldots & 0 & 1\\
			 1  & -2 & 1 &             & 0 & 0 \\
			 0 & 1 & -2 &      \ddots        &   &  \\
			    &    &    \ddots  &      \ddots        &   &  \\
			0 & 0 &   &                & -2 & 1\\
			1 & 0 &  \ldots &  & 1 & -2\\
		 \end{pmatrix}
\quad\text{  et  }\quad
	D_4 = \begin{pmatrix}
			 6 & -4  & 1 & 0 &\ldots & 1 & -4\\
			 -4  & 6 & -4 & \ddots  &          & 0 & 1 \\
			 1 & -4 & 6 & \ddots   &   \ddots       &   & 0  \\
			  0  &  \ddots  &   \ddots   &  \ddots   &   \ddots      &   &  \\
			    &    &  \ddots    &  \ddots   &    \ddots     &   & 0 \\
			  0  &    &   \ddots   &  \ddots   &         &   & 1 \\
			1 & 0 &   &      &          & -6 & -4\\
			-4 & 1 &  0 & \ldots & 1 & -4 & 6\\
		 \end{pmatrix}
\end{displaymath}
La discr\'{e}tisation temporelle se fait simplement en prenant la diff\'{e}rence entre deux instants : 
\begin{displaymath}
	\frac{\partial \gamma}{\partial t} \simeq \frac{\gamma^{k}-\gamma^{k-1}}{\Delta t}
\end{displaymath}
Au total, on trouve pour chaque composante les \'{e}quations : 
\begin{eqnarray}~\label{discreteSnake}
	\frac{\gamma_{x}^{k}-\gamma_{x}^{k-1}}{\Delta t}=& (\lambda_1 D_2 -\lambda_2 D_4)\gamma_{x}^{k} + \lambda_3 \nabla(\|\nabla \text{I}\|^2)_{x}\\
	\frac{\gamma_{y}^{k}-\gamma_{y}^{k-1}}{\Delta t}=& (\lambda_1 D_2 -\lambda_2 D_4)\gamma_{y}^{k} + \lambda_3 \nabla(\|\nabla \text{I}\|^2)_{y}
\end{eqnarray}
avec $\nabla(\|\nabla \text{I}\|^2)_{x}$ et $\nabla(\|\nabla \text{I}\|^2)_{y}$ les composantes selon $x$ et $y$ de $\nabla(\|\nabla \text{I}\|^2)$. Le syst\`{e}me d'\'{e}volution liant les deux instants $k$ et $k-1$ peut s'\'{e}crire : 
\begin{eqnarray}\label{evolutionSnakex}
	(I_N+\Delta t (\lambda_2 D_4 - \lambda_1 D_2)) \gamma_{x}^{k} =&  (\gamma_{x}^{k-1} + \Delta t  \lambda_3 \nabla(\|\nabla \text{I}\|^2)_{x} )\\
~\label{evolutionSnakey}
	(I_N+\Delta t (\lambda_2 D_4 - \lambda_1 D_2)) \gamma_{y}^{k} =&  (\gamma_{y}^{k-1} + \Delta t  \lambda_3 \nabla(\|\nabla \text{I}\|^2)_{y} )
\end{eqnarray}
avec \textbf{$I_N$ la matrice identit\'{e} d'ordre $N$}. On obtient donc un syst\`{e}me lin\'{e}aire pour chaque composante de la courbe.

%\subsection{Les ameliorations possible}
%ballon (aborde dans le TP), gradient vector flow.
%Autre : level set (probleme de topologie qui change)


\clearpage
\section{Pratique}
\subsection{Les outils}
Pour ce TP, on vous demande de r\'{e}alise les calculs en C/C++ et la visualisation avec \textit{octave}.

Nous mettons \`{a} votre disposition une classe \textit{C\_matrix.h} pour g\'{e}rer les matrices et faire des op\'{e}rations simples comme l'addition, la soustraction, la multiplication, ainsi que des op\'{e}rations plus complexes comme l'inversion de matrice, la convolution, le calcul de gradient...

\paragraph{Exemple 1}
Exemple de code pour la cr\'{e}ation d'une matrice: 
\begin{verbatim}
#include <C_matrix.h>//gestion des operations matricielles
#include <iostream>

int main(int argc, char *argv[])
{
    //creation d'un objet
    C_matrix<double> my_matrix(3,5);
    //initialisation random de la matrice
    my_matrix.randomf();
   //affichage sur la console
    my_matrix.show();
    //sauvegarde de l'objet dans un fichier ``myMatrixFile''
    my_matrix.save("myMatrixFile");

    //affichage de la somme de tout les elements de my_matrix
    std::cout << my_matrix.sum() << std::endl;

    return 1;
}
\end{verbatim}

\paragraph{Exemple 2} Pour le chargement et l'affichage d'image, la classe \textit{C\_imgMatrix} qui h\'{e}rite de la classe \textit{C\_matrix} utilise la librairie \textit{CImg} pour charger et afficher les images. La librairie \href{http://cimg.sourceforge.net/reference/group__cimg__tutorial.html}{\textit{CImg}} peut \^{e}tre install\'{e}e sur votre ordinateur avec la commande 
\begin{center}
	\textit{apt-get install cimg-dev}
\end{center}
ou, si vous n'avez pas les droits n\'{e}cessaires, vous pouvez cloner le repository git de \textit{CImg} en tapant la commande suivante : 
\begin{center}
	\textit{git clone http://git.code.sf.net/p/cimg/source CImg}
\end{center}
Remarque : \textit{git} est un gestionnaire de version. Les gestionnaires de versions servent \`{a} g\'{e}rer les versions d'un projet et de pouvoir revenir \`{a} touts moments \`{a} une version pr\'{e}c\'{e}dente pour pouvoir retrouver un code/version qui fonctionne. Je vous recommande fortement d'utiliser des gestionnaires de versions dans tout ce que vous faites. Par exemple, vous commencez un TP, vous cr\'{e}ez un repository que vous actualiserez tout au long de la s\'{e}ance de sorte \`{a} ce que vous puissiez retrouver une version qui fonctionne ou qui fait autre chose (une autre partie d'un exercice). Pour plus d'information concernant \textit{git}, je vous renvoie vers le \href{https://github.com/}{lien} ou un autre \href{http://rogerdudler.github.io/git-guide/}{lien}.

\noindent \textbf{Remarque : la version actuelle de \textit{CImg} \'{e}tant diff\'{e}rente de celle utilis\'{e}e dans le code de la classe \textit{C\_imgMatrix}, le fichier \textit{CImg.h} correspondant \`{a} la version utilis\'{e}e dans le code \`{a} \'{e}t\'{e} mis sur le repository git du TP.}

\clearpage
\noindent Exemple de code pour charger une image \`{a} l'aide de la classe \textit{C\_imgMatrix} : 
\begin{verbatim}
#include <C_imgMatrix.h> //inclu la library de chargement d'image CImg
#include <iostream>

int main(int argc, char *argv[])
{
    if(argc!=2)
    {
        std::cout << "Commande : " << argv[0] << " fileName<string>" << std::endl;
        return -1;
    }
    //charge l'image argv[1] dans la matrice myImg
    C_imgMatrix<double> myImg(argv[1]);
    //montre l'image
    myImg.display(0);
    //calcul le gradient dans la direction X
    C_imgMatrix<double> myGradX = myImg.gradX();
    //affiche l'image du gradient
    myGradX.display(1);
    return 1;
}
\end{verbatim}

\subsection{Questions}
\subsubsection{Charger et afficher une image}
Tester le chargement d'une image avec la classe \textit{C\_imgMatrix} et afficher votre image.
\subsubsection{Effectuer des op\'{e}rations simple}
Filtrez une image par un noyau gaussien (Attention! Le noyau doit \^{e}tre de dimension impaire). Affichez le r\'{e}sultats. Affichez la valeur absolue de la diff\'{e}rence entre l'image d'origine et l'image filtr\'{e}e. Sur une image binaris\'{e}e, utilisez la m\'{e}thode \textit{bwdistEuclidean} pour calculer la carte de distance, la visualiser.

\subsection{G\'{e}n\'{e}ration de deux courbes}
Pour des soucis de simplicit\'{e}s conceptuelles, nous utiliserons des objets qui encoderont les points et les courbes, respectivement les classes \textit{C\_point} et \textit{C\_curve}. La classe \textit{C\_point} contient des attributs de position \textit{x} et \textit{y}, de vitesse \textit{vx} et \textit{vy} ainsi que d'\'{e}nergie image \textit{Px} et \textit{Py}. La classe \textit{C\_curve} contient un vecteur de point ainsi que deux m\'{e}thodes pour initialiser la positions des points de la courbe, une en forme de cercle et l'autre en forme de carr\'{e}. Regardez rapidement de quoi sont faites ces classes avant de commencer.
Dans la classe \textit{C\_curve}, cr\'{e}ez deux m\'{e}thodes pour extraire les coordonn\'{e}es x et y de la courbe et les stocker dans deux matrices colonnes. Cr\'{e}ez une matrice pour calculer la d\'{e}riv\'{e}e de la courbe, calculez la d\'{e}riv\'{e}e et sauvegardez les r\'{e}sultats dans des fichiers. Calculez de l'\'{e}nergie int\'{e}rieure de la courbe au sens des snakes. Sous octave, chargez les fichiers et les afficher \`{a} l'aide des fonctions \textit{load} et \textit{plot}.

\subsection{\'{E}nergie int\'{e}rieure du Snake}
Dans la classe \textit{C\_snakes}, dont un squelette est propose, impl\'{e}mentez deux m\'{e}thodes qui g\'{e}n\`{e}rent les matrices de d\'{e}rivation $D1$ et d\'{e}rivation seconde $D2$. Les deux matrices $D1$ et $D2$ sont des attributs de la classe. Cr\'{e}ez ensuite une m\'{e}thode qui r\'{e}alise le calcul de l'\'{e}nergie interne de la courbe.

\subsection{Carte d'\'{e}nergie ext\'{e}rieure}
Dans la classe \textit{C\_snakes}, cr\'{e}ez une m\'{e}thode qui calcule l'\'{e}nergie image en chaque point de l'image, ainsi que les composantes du gradient de $E_{\text{ext}}$ (\'{e}nergie image). On stockera l'\'{e}nergie image dans \textit{imgEnergy} et les composantes du gradient dans les attributs \textit{PX} et \textit{PY}.

%image generale : gradient de la norme du gradient de l'image
%image de trait : l'image elle meme (ecriture foncee sur page blanche)
%image binaire : transformee de distance.

\subsection{\'{E}volution temporelle du snake}
Cr\'{e}ez une m\'{e}thode qui g\'{e}n\`{e}re la matrice de d\'{e}riv\'{e}e $4^{\text{\`eme}}$ $D4$. Impl\'{e}mentez une m\'{e}thode qui g\'{e}n\`{e}re le syst\`{e}me d\'{e}fini \`{a} l'\'{e}quation~\eqref{evolutionSnakex} et~\eqref{evolutionSnakey}  et qui stocke l'inversion du syst\`{e}me dans la matrice Ainv. Cr\'{e}ez une methode qui met \`{a} jour les attributs Px et Py de chaque points de la courbe. Impl\'{e}mentez deux m\'{e}thodes qui retournent les seconds membres des \'{e}quations~\eqref{evolutionSnakex} et~\eqref{evolutionSnakey}. Pour faire \'{e}voluer la courbe depuis un point de d\'{e}part fixe, d\'{e}finissez une m\'{e}thode qui initialise tout les \'{e}l\'{e}ments n\'{e}cessaire \`{a} l'\'{e}volution, puis cr\'{e}ez une m\'{e}thode qui encode un pas temporel de l'\'{e}volution du snake. Enfin, rassemblez tout les \'{e}l\'{e}ments cod\'{e}s auparavant de sorte \`{a} ce qu'une seule m\'{e}thode soit appel\'{e}e dans le \textit{main}. 

Testez votre code sur les images qui vous sont fournies et d\'{e}terminez les param\`{e}tres $\lambda_1$, $\lambda_2$, $\lambda_3$ et $\mathrm{d}t$ pour ces images. Visualisez les effets de chaque param\`{e}tre pour les diff\'{e}rentes images. Montrez l'\'{e}volution du snake et de l'\'{e}nergie totale en fonction du temps.

sur les diff\'{e}rentes images, faire \'{e}voluer le snake et trouver des bons param\`{e}tres pour trouver les contours d\'{e}sir\'{e}s.

\subsection{Pour aller plus loin}
Si vous avez le temps pendant la s\'{e}ance de TP, vous pouvez explorer les pistes suivantes : 
\begin{itemize}
	\item[1] Changez la forme de l'\'{e}nergie int\'{e}rieure
	\item[2] Appliquez une force de pression sur le snake en ajoutant une force ext\'{e}rieure selon la normale \`{a} la courbe. Vous pouvez vous referez \`{a} l'article \href{http://ac.els-cdn.com/104996609190028N/1-s2.0-104996609190028N-main.pdf?_tid=8cc41376-c1ff-11e4-a8e9-00000aab0f26&acdnat=1425426585_a1dae6c4928c4fa3bc99d9bffe5d920c}{On active contour models and ballons}
	\item[3] Pour arriver \`{a} detecter les contours dans une forme concave, vous pouvez utiliser une technique qui consiste \`{a} faire \'{e}voluer le snake dans un champ de vecteur. Vous pouvez vous r\'{e}f\'{e}rer \`{a} la publication \href{http://www-artemis.it-sudparis.eu/~rougon/IMA4509/Controle/Sujets/Xu-IP-1998.pdf}{Snakes, shapes, and gradient vector flow}
\end{itemize}
%essaye de faire evoluer le snake sur les contour de l'objet
%Pour eviter de rester coincer sur de petites aretes non significative, on decide d'introduire une force de pression exterieur qui est appliquee sur la normale de la courbe













\end{document}





